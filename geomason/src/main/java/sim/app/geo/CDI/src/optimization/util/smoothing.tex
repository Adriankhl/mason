\documentclass[10pt]{article}
\usepackage{url,graphicx,tabularx,amsmath,amsfonts,fullpage,multicol,makecell}
\usepackage{verbatim}
% \usepackage{subfigure}
\usepackage{caption}
\usepackage{subcaption}
\usepackage{float}
\usepackage{algorithmicx}
\usepackage{algpseudocode}
\usepackage{tikz}
\usetikzlibrary{automata,positioning}

\setlength{\parskip}{1ex} %--skip lines between paragraphs
\setlength{\parindent}{0pt} %--don't indent paragraphs
%-- Commands for header
\renewcommand{\title}[1]{\textbf{#1}\\}
\renewcommand{\line}{\begin{tabularx}{\textwidth}{X>{\raggedleft}X}\hline\\\end{tabularx}\\[-0.5cm]}
\newcommand{\leftright}[2]{\begin{tabularx}{\textwidth}{X>{\raggedleft}X}#1%
& #2\\\end{tabularx}\\[-0.5cm]}

%----- my mod
\makeatletter
%-- change section and subsection title format
\renewcommand\section{\@startsection{section}{1}{\z@}%
                                  {-3.5ex \@plus -1ex \@minus -.2ex}%
                                  {.05ex \@plus.05ex}%
                                  {\normalfont\large\bf}}
\renewcommand\subsection{\@startsection{subsection}{2}{\z@}%
                                  {-3.5ex \@plus -1ex \@minus -.2ex}%
                                  {.05ex \@plus.05ex}%
                                  {\normalfont\normalsize\bf}}
%-- put "." (dot) after each section number
% \g@addto@macro\thesection.
% \makeatother

%\linespread{2} %-- Uncomment for Double Space

\begin{document}

\title{Smoothing the landscape by Gaussian Kernel}
\line
\leftright{\vspace{0.5pt}\today}{\vspace{0.5pt}AKM Khaled Ahsan Talukder} %-- left and right positions in the header

\vspace{10pt}
Hi, I have done with the smoothing code. I made it generic so that it could be used with any library (i.e. MASON's grid), you need to just pass the grid as a double array. The smoothing is done with Gaussian kernel (i.e. Gaussian Blur \url{http://en.wikipedia.org/wiki/Gaussian_blur}). The complexity is $O(n^2)$.
%
\begin{figure}[H]
\centering
	\begin{subfigure}[b]{0.5\textwidth}
		\includegraphics[width=\textwidth]{orig-terrain.pdf}
		\caption{Original Terrain}
	\end{subfigure}%
	\begin{subfigure}[b]{0.5\textwidth}
		\includegraphics[width=\textwidth]{smooth-terrain-0.pdf}
		\caption{Iteration 0}
	\end{subfigure}%
\end{figure}
\begin{figure}[H]
	\begin{subfigure}[b]{0.5\textwidth}
		\includegraphics[width=\textwidth]{smooth-terrain-1.pdf}
		\caption{Iteration 1}
	\end{subfigure}
	\begin{subfigure}[b]{0.5\textwidth}
		\includegraphics[width=\textwidth]{smooth-terrain-2.pdf}
		\caption{Iteration 2}
	\end{subfigure}
\end{figure}
\begin{figure}[H]
	\begin{subfigure}[b]{0.5\textwidth}
		\includegraphics[width=\textwidth]{smooth-terrain-3.pdf}
		\caption{Iteration 3}
	\end{subfigure}
	\begin{subfigure}[b]{0.5\textwidth}
		\includegraphics[width=\textwidth]{smooth-terrain-4.pdf}
		\caption{Iteration 4}
	\end{subfigure}
\end{figure}
\begin{figure}[H]
	\begin{subfigure}[b]{0.5\textwidth}
		\includegraphics[width=\textwidth]{smooth-terrain-5.pdf}
		\caption{Iteration 5}
	\end{subfigure}
	\begin{subfigure}[b]{0.5\textwidth}
		\includegraphics[width=\textwidth]{smooth-terrain-6.pdf}
		\caption{Iteration 6}
	\end{subfigure}
\end{figure}
\begin{figure}[H]
	\begin{subfigure}[b]{0.5\textwidth}
		\includegraphics[width=\textwidth]{smooth-terrain-7.pdf}
		\caption{Iteration 7}
	\end{subfigure}
	\begin{subfigure}[b]{0.5\textwidth}
		\includegraphics[width=\textwidth]{smooth-terrain-8.pdf}
		\caption{Iteration 8}
	\end{subfigure}
\end{figure}
\begin{figure}[H]
	\begin{subfigure}[b]{0.5\textwidth}
		\includegraphics[width=\textwidth]{smooth-terrain-9.pdf}
		\caption{Iteration 9}
	\end{subfigure}
	\begin{subfigure}[b]{0.5\textwidth}
		\includegraphics[width=\textwidth]{smooth-terrain-10.pdf}
		\caption{Iteration 10}
	\end{subfigure}
\end{figure}
%
\end{document}
